%%%%%%%%%%%%%%%%%%%%%%%%%%%%%%%%%%%%%%%%%%%%%%%%%
%%%%%%%%%%%% chap: title %%%%%%%%%%%%%%%%%
%%%%%%%%%%%%%%%%%%%%%%%%%%%%%%%%%%%%%%%%%%%%%%%%%

\chapter{State of the art}\label{chapter:chap2}

First step is  to search what has be done in field your thesis.

\section{Resources}

Scientific articles can be found in different online database:
\begin{enumerate}
    \item ACM Digital Library
    \item Annual Reviews
    \item IEEE Explore
    \item JSTOR (social sciences, arts \& humanities)
    \item Science Direct
    \item Springer Link
    \item Wiley InterScience
    \item SCOPUS
    \item  Google Scholar
\end{enumerate}

Identify communities on the topic (specific workshops on national and international conferences)

Some of the articles are accessible through university IP.

Some other thesis on the topic an be consulted: \url{oatd.org}, \url{openthesis.org}, etc.

\section{Take notes}

For each read paper try to note information, so you can use later:
\begin{itemize}
    \item resource author;
    \item resource name (article title, book title, ...);
    \item subject: what is the paper aim;
    \item addressed problem ; 
    \item methods/methodology used to solve the problem;
    \item used algorithms;
    \item test data (e.g. benchmarks, real cases).
\end{itemize}

For a git resource or an existing application:
\begin{itemize}
    \item URL;
    \item application features;
    \item used technologies.
\end{itemize}