\documentclass[12pt,a4paper]{book}
\newcommand\tab[1][1cm]{\hspace*{#1}}
\usepackage{amsmath,amsthm,amssymb,graphicx,hyperref}
\usepackage[left=1.2in,right=1.2in,top=1in,bottom=1in]{geometry}
\usepackage{listings}
%\usepackage[romanian]{babel}
\usepackage[english]{babel}
\usepackage{xcolor}
\usepackage{paralist}
\RequirePackage{hyphenat}
\usepackage{algorithm, algpseudocode}%algorithmic
\usepackage[inline]{enumitem}
%\usepackage{...insert other packages here...}
\newtheorem{thm}{Theorem}[section]
\newtheorem{lem}[thm]{Lemma}
\theoremstyle{definition}
\newtheorem{defn}{Definition}[section]
\theoremstyle{remark}
\newtheorem{rem}{Remark}[section]
\newtheorem{exmp}{Example}[section]


\begin{document}
\sloppy

\thispagestyle{empty}
\begin{center}
\begin{figure}[h!]
\vspace{-20pt}
\begin{center}
\includegraphics[width=100pt]{FMI-03.png}
\end{center}
\end{figure}


{\large{\bf WEST UNIVERSITY OF TIMI\c SOARA

FACULTY OF MATHEMATICS AND COMPUTER SCIENCE

BACHELOR STUDY PROGRAM:  COMPUTER SCIENCE IN ENGLISH  
}}

\vspace{120pt}
{\huge {\bf BACHELOR THESIS}}

\vspace{160pt}
\end{center}

{\large\noindent{\bf SUPERVISOR:
\hspace{172pt} GRADUATE: }

\noindent Prof./Conf./Lect. Dr. Firstname Lastname \hfill 
\noindent  Firstname Lastname
}

\vfill
\begin{center}
{\bf TIMI\c SOARA

2025}
\end{center}
\newpage
\thispagestyle{empty}
\begin{center}
{\large{\bf WEST UNIVERSITY OF TIMI\c SOARA
		
FACULTY OF MATHEMATICS AND COMPUTER SCIENCE
		
BACHELOR STUDY PROGRAM:  COMPUTER SCIENCE IN ENGLISH}}

\vspace{200pt}
{\huge {\bf Title }}

\vspace{153pt}
\end{center}

{\large\noindent{\bf SUPERVISOR:\hfill GRADUATE:}

\noindent Prof./Conf./Lect. Dr. Firstname  Lastname\hfill
\noindent Firstname  Lastname}
 

\vfill
\begin{center}
{\bf TIMI\c SOARA

2025}
\end{center}

\newpage
\normalsize{}

\section*{Abstract} 


\newpage
\normalsize{}

\tableofcontents
\input{cap1} % example
%%%%%%%%%%%%%%%%%%%%%%%%%%%%%%%%%%%%%%%%%%%%%%%%%
%%%%%%%%%%%% chap: title %%%%%%%%%%%%%%%%%
%%%%%%%%%%%%%%%%%%%%%%%%%%%%%%%%%%%%%%%%%%%%%%%%%

\chapter{State of the art}\label{chapter:chap2}

First step is  to search what has be done in field your thesis.

\section{Resources}

Scientific articles can be found in different online database:
\begin{enumerate}
    \item ACM Digital Library
    \item Annual Reviews
    \item IEEE Explore
    \item JSTOR (social sciences, arts \& humanities)
    \item Science Direct
    \item Springer Link
    \item Wiley InterScience
    \item SCOPUS
    \item  Google Scholar
\end{enumerate}

Identify communities on the topic (specific workshops on national and international conferences)

Some of the articles are accessible through university IP.

Some other thesis on the topic an be consulted: \url{oatd.org}, \url{openthesis.org}, etc.

\section{Take notes}

For each read paper try to note information, so you can use later:
\begin{itemize}
    \item resource author;
    \item resource name (article title, book title, ...);
    \item subject: what is the paper aim;
    \item addressed problem ; 
    \item methods/methodology used to solve the problem;
    \item used algorithms;
    \item test data (e.g. benchmarks, real cases).
\end{itemize}

For a git resource or an existing application:
\begin{itemize}
    \item URL;
    \item application features;
    \item used technologies.
\end{itemize}

\chapter{Some Latex Statements}

In this chapter are presented  some useful latex statements in thesis written part.


\section{Enumeration usage}
Enumeration with numeric label:
\begin{enumerate}
    \item Option 1
    \item Option 2
    \item Option 3
\end{enumerate}

Enumeration without label:
\begin{itemize}
    \item Option 1
    \item Option 2
    \item Option 3
\end{itemize}

\section{If you need to add: definitions,  theorems, ... .}
\begin{defn} ...
\end{defn}
\begin{thm} ...
\end{thm}
\begin{proof} ...
\end{proof}
\begin{rem} ...
\end{rem}
\begin{exmp} ...
\end{exmp}

\section{References}
The bibliography has to be referenced in thesis content using cite (e.g. \cite{Bersani}).

\section{Using figures}
Each figure has to have a caption that is a suggestive description of what  the  picture represents (e.g. Figure \ref{fig:siglaUVT}).
\begin{figure}[!ht]
    \centering
    \includegraphics[width=0.25\linewidth]{template-eng/FMI-03.png}
    \caption{ FMI logo scaled at 25\% of text width}
    \label{fig:siglaUVT}
\end{figure}

\section{Algorithm pseudo-code}
Pseudo-code is a formal way to describe an algorithm, is more clear than a textual description ore code ( e.g. Algorithn \ref{alg:alg_ex}).

\begin{algorithm}[!ht]
\caption{An algorithm with caption}\label{alg:alg_ex}
\begin{algorithmic}[1]
\Require $n \geq 0$
\Ensure $y = x^n$
\State $y \gets 1$
\State $X \gets x$
\State $N \gets n$
\While{$N \neq 0$}
\If{$N$ is even}
    \State $X \gets X \times X$
    \State $N \gets \frac{N}{2}$  \Comment{This is a comment}
\ElsIf{$N$ is odd}
    \State $y \gets y \times X$
    \State $N \gets N - 1$
\EndIf
\EndWhile
\end{algorithmic}
\end{algorithm}

\section{Adding code}
If it necessary to add  some code from the application, do not use print-screens, use  listing  (e.g. listing \ref{lst:p1}).
\begin{lstlisting}[language=Python, numbers=left,
    stepnumber=1, caption=Un exemplu de cod python, label=lst:p1]
import numpy as np
    
def incmatrix(genl1,genl2):
    m = len(genl1)
    n = len(genl2)
    M = None #to become the incidence matrix
    VT = np.zeros((n*m,1), int)  #dummy variable
    
    #compute the bitwise xor matrix
    M1 = bitxormatrix(genl1)
    M2 = np.triu(bitxormatrix(genl2),1) 

    for i in range(m-1):
        for j in range(i+1, m):
            [r,c] = np.where(M2 == M1[i,j])
            for k in range(len(r)):
                VT[(i)*n + r[k]] = 1;
                VT[(i)*n + c[k]] = 1;
                VT[(j)*n + r[k]] = 1;
                VT[(j)*n + c[k]] = 1;
                
                if M is None:
                    M = np.copy(VT)
                else:
                    M = np.concatenate((M, VT), 1)
                
                VT = np.zeros((n*m,1), int)
    
    return M
\end{lstlisting}

\section{Tablels}

A simple example of an table (see Table \ref{tab:my_tabel}). 

\begin{table}[!ht]
    \centering
    \begin{tabular}{|l | c |  c | c |} 
 \hline
 Stopping criteria & \textbf{Alg.1} & \textbf{Alg.2} & \textbf{Alg.3} \\  
 \hline\hline
\textbf{ MSQ} & 0.97 & 0.8 & 00.60 \\ 
 \hline
 \textbf{R2} & 0.77 & 0.78 & 0.54 \\
 \hline
 
 \hline
\end{tabular}
    \caption{Algorithm comparison}
    \label{tab:my_tabel}
\end{table}


For more details regarding how to create a table use the following reference \url{https://www.overleaf.com/learn/latex/Tables}.


\bibliography{mybib}
\bibliographystyle{alpha}
\addcontentsline{toc}{chapter}{Bibliography}

\end{document} 